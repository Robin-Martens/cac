\documentclass{article}
\usepackage[utf8]{inputenc}
\usepackage{graphicx}
\usepackage{booktabs}
\usepackage[margin=1in]{geometry}
\usepackage{url}
\usepackage[toc,page]{appendix}
\usepackage{lscape}
\usepackage{subcaption}
\usepackage{longtable}
\usepackage[hidelinks]{hyperref}
\usepackage{biblatex} %Imports biblatex package
\usepackage{fancyhdr}
\usepackage{lastpage}
\usepackage{framed}
\usepackage{pgfplots}
\usepackage[T1]{fontenc}
\usepackage{subcaption}
\usepackage{amsmath, amsthm, amssymb, amsfonts}
\newcommand{\textbox}[2][6]{
\begin{framed}
\noindent#2
\end{framed}}

\addbibresource{bibliography.bib} %Import the bibliography file \title{Threat Analysis: a Patient Monitoring System}
\author{Robin Martens}
\date{19/08/2023}

\newcommand{\tabitem}{\textbullet \ }
\newcommand{\https}{\textsc{https}}
\newcommand{\smtp}{\textsc{smtp}}
\newcommand{\threat}[4]{#1 & \textit{#2} & #3 & #4 \\}

\renewcommand{\sectionmark}[1]{\markright{\thesection.\ #1}}

\fancyhead[L]{Robin Martens: r0885874}
\fancyhead[C]{}
\fancyhead[R]{}
\fancyfoot[L]{Report Project II}
\fancyfoot[C]{}
\fancyfoot[R]{\thepage /\pageref{LastPage}}

\renewcommand{\headrulewidth}{0.4pt}
\renewcommand{\footrulewidth}{0.4pt}

\definecolor{s}{HTML}{25282a}
\definecolor{t}{HTML}{003b56}
\definecolor{r}{HTML}{0d514c}
\definecolor{i}{HTML}{0f471f}
\definecolor{d}{HTML}{473300}
\definecolor{e}{HTML}{522000}

\newcounter{assumptioncounter}
\DeclareRobustCommand{\assumption}{%
   \refstepcounter{assumptioncounter}%
   \textit{AS\theassumptioncounter :}~}

\begin{document}
\begin{titlepage}
    \newpage
    \pagenumbering{Roman}
    \thispagestyle{empty}
    \frenchspacing
    \hspace{-0.2cm}
    \hspace{0.2cm}
    \hspace{0.2cm}
    \includegraphics[height=2.5cm]{images/misc/logoFirW.jpg}
    \hspace{5.8cm}
    \includegraphics[height=2.5cm]{images/misc/kul_logo.jpg}
    \hspace{\stretch{1}}
    \vfill
    \vspace{0.5cm}
    \centering\Large{\rm\textbf{{Computer Algebra for Cryptography}}}
    \vspace{5cm}
    \begin{center}
        \begin{minipage}[t]{\textwidth}
            \begin{center}
                \Huge{\rm\textbf{Project I}} \\
                \vspace{0.5cm}
                \large{\rm\textsc{{final report}}} \\
                \vspace{1cm}
                {\rm {Robin Martens}} \\
                \vspace{9cm}
                \LARGE{\rm\textsc{academic year 2024-2025}}
                \vfill
            \end{center}
        \end{minipage}
    \end{center}
    \vfill
\end{titlepage}


\pagenumbering{arabic}
\setcounter{page}{1}
\pagestyle{fancy}


\subsection*{Task 1}
\subsubsection*{Task 1.b}

We first note that $\left| \mathbb{F}_N^n \right|  = N^n$ and that every
linear disequation $a \cdot s \neq b$ deletes  $\frac{1}{N}$ from the solution
space, so there remain $\frac{N - 1}{N} \cdot N^n$ solutions after 1
disequation. After $m$ disequations, we have $N^n\left(\frac{N-1}{N} \right)^m 
$ remaining solutions in the solution space, so we want that amount to be equal
to $1$. Solving that equation for $m$ yields:
\begin{equation}
m = \left( \frac{\ln N}{- \ln(1 - \frac{1}{N})} \right) n,
\end{equation} 
such that $C_N = -\frac{\ln N}{\ln(1 - \frac{1}{N})}$.

\subsubsection*{Task 1.c}

As $N$ is a prime, we know that $\varphi(N) = N - 1$ and from above we have that
the number of solutions after $m$ disequations is given by 
\begin{equation}
  N^{n - m} (N - 1)^m = N^{n - m} \varphi(N)^{m - 1},
\end{equation} 
such that the statement in this question is true.

% TODO: what about zero solutions??

\subsection*{Task 2}
\subsubsection*{Task 2.a}

In $\mathbb{F}_2$, there are only two possible coefficients for $
\mathbf{a}$, $\mathbf{s}$ and $b$, so simply inverting $b$ will transform
the disequation into an equation. For example, if we have an $\mathbf{s}$ which
complies with 
\begin{equation}
a_1 s_1 + \dots + a_n s_n \neq b
\end{equation} 
so the \textsc{lhs} is either $0/1$ and the \textsc{rhs} is $1/0$.
then the same $\mathbf{s}$ will comply with 
\begin{equation}
  a_1 s_1 + \dots + a_n s_n = \overline{b}
\end{equation} 
as now the \textsc{lhs} and \textsc{rhs} are equal (we can also invert the
\textsc{lhs}, so $\mathbf{a}$).

\subsubsection*{Task 2.b}

We first repeat Fermat's little theorem: for an $a \in \mathbb{F}_p$ with
$\gcd(a, p) = 1$:
\begin{equation}
  a^{p-1} = 1
\end{equation} 
and thus, as $p$ is a prime, $\gcd(a, p) = 1$ if and only if $a \neq 0$. With
Fermat's little theorem, we can transform $a \neq 0$ into $a^{p - 1} = 1$.
Starting from the equation 
\begin{equation}
  \begin{split}
    a_1 x_1 + \dots + a_n x_n \neq b &\Leftrightarrow a_1 x_1 + \dots + a_n x_n - b
    \neq 0 \\
                                     &\Leftrightarrow  (a_1 x_1 + \dots + a_n
                                     x_n - b)^{p-1} = 1
  \end{split}
\end{equation} 

and thus function $f$ is $f(x_1, \dots, x_n) = (a_1x_1 + \dots x_n a_n -b)^{p-1}
- 1$.

\clearpage

\subsubsection*{Task 2.c}

The table below summarizes the results for different values of $r$ and $p$.

\begin{table}[htpb]
  \centering
  \caption{Timing results for $n = 8$.}
  \label{tab:label}
  \begin{tabular}{cccc}
  
    \toprule
    $r$ & $p$ & Time taken [s]  & \# eqs.\\
    \midrule
    1 & 2 & 0.000 & 8\\
    \midrule
    1 & 3 & 0.010 & 22\\
    \midrule
    1 & 5 & 0.540 & 58\\
    \midrule
    1 & 7 & 218.580 & 101\\
    \midrule
    2 & 2 & 0.000 & 16 \\
    \midrule
    2 & 3 & 0.010 & 44 \\
    \midrule
    2 & 5 & 0.120 & 116\\
    \midrule
    2 & 7 & 39.860 & 202\\
    \midrule
    3 & 2 & 0.000 & 24\\
    \midrule
    3 & 3 & 0.010 & 66 \\
    \midrule
    3 & 5 & 0.090 & 174 \\
    \midrule
    3 & 7 & 21.930 & 303\\
    \bottomrule
  \end{tabular}
\end{table}

We can see that adding the field equations does not have an influence on the
runtime of the algorithm, which is logical, given the fact that the field
equations actually just state Fermat's little theorem and as $p$ is prime, the
field equation will hold for any member of the field $\mathbb F_p$.

\begin{table}[htpb]
  \centering
  \caption{Timing results with added field equations and $n = 8$.}
  \label{tab:label}
  \begin{tabular}{ccc}
   \toprule
   $p$ & Time taken [s] & \# eqs. \\
   \midrule
   2 & 0.000 & 16 \\
   \midrule
   3 & 0.000 & 30 \\
   \midrule
   5 & 0.570 & 66 \\
   \midrule
   7 & 217.220 & 109 \\
   \bottomrule
  \end{tabular}
\end{table}

\subsubsection*{Task 2.d}

Euler's congruence states that for an $a \in \mathbb{F}_N$ and $\gcd(a, N) = 1$,
$a^{\phi(N)} = 1$. So now, we cannot transform the disequation $a_1 x_1 + \dots
+ a_n x_n \neq 0$ simply into an equation, as $a_1 x_1 + \dots + a_n x_n$ can be
equal to a multiple of $p$, such that Euler's congruence does not work anymore.

\subsection*{Task 3}
\subsubsection*{Task 3.a}

Fermat's little theorem states that for an $a \in \mathbb{F}_p$, $a^{p-1} = 1$
if $a \neq 0$ and $a^{p - 1} = 0$ if $a = 0$. Then we define, $f(x) = 1 -
\left( x -  c \right)^{p - 1}$, such that $f(x) = 1$ if  $x = c$ and $f(x) = 0$
if $x \neq  c$. Extending this to $c$ and $d$ yields: 
\begin{equation}
  f(x, y) = \left( 1 - (x - c)^{p-1} \right) \left( 1 - (y - d)^{p - 1} \right),
\end{equation} 
such that $f(x, y) = 1$ if $(x, y) = (c, d)$ and $f(x,y) = 0$ if $(x, y) \neq
(c, d)$. Now, as stated in the assignment, $W(x, y)$ is now the sum of the above
polynomials when $c + d \ge  p$ : 
\begin{equation}
  W(x, y) = \sum_{\substack{c, d \in \mathbb{F}_p \\ c + d \ge p}}
\left( 1 - (x - c)^{p-1} \right) \left( 1 - (y - d)^{p - 1} \right).
\end{equation} 

\subsubsection*{Task 3.b}
We have $c = c_0 + c_1p$ and $d = d_0 + d_1p$, such that $c + d = (c_0 + c_1p) +
(d_0 + d_1p) = (c_0 + d_0) + (c_1 + d_1)p$. Now, if $c_0 + d_0 \ge  p$, we have
to add $1$ to the coefficient of $p$, which is $W(c_0, d_0)$. If $c_1 + d_1 \ge
p$, we have to reduce modulo $p$, as the result $c + d$ will also be reduced
modulo $p^2$.

\subsubsection*{Task 3.c}

We first find the functions $f_1$ and $f_2$ from the hint. For that, we write
all $a_i$, $x_i$ and $b$ as: $a_i = a_{i0} + p a_{i1}$, $x_i = x_{i0} + p
x_{i1}$ and $b = b_0 + p b_1$. We fill this into the expression  $a_1 x_1 +
\dots + a_n x_n - b$ and after multiplying out and reducing $\text{mod}\ p^2$:

\begin{equation}
\begin{cases}
  f_0(x_{11}, \dots, x_{n2}) = a_{10} x_{10} + \dots + a_{n0} x_{n0} - b_0 \\
  f_1(x_{11}, \dots, x_{n 2}) = a_{11} x_{10} + a_{10}x_{11} + \dots + a_{n_1}
  x_{n0} + a_{n 0} x_{n 1} - b_1
\end{cases}.
\end{equation} 
Now, it is still possible that $f_0(s) \ge p$, in which case we would need to add
something to $f_1$. We can use the $W$-function for this, where we have to
decompose $f_0$ to sums of elements in $\mathbb F_p$. First, we note that 
\begin{equation}
  \begin{cases}
    2 x_{10} = \overline{2 x_{10}} + W(x_{10}, x_{10}) \\
    3 x_{10} = 2x_{10} + x_{10} = \overline{3 x_{10}} + W(x_{10}, x_{10}) +
    W(2x_{10}, x_{10})
  \end{cases},
\end{equation} 
where the $\overline{\cdot}$ indicates that the result is $< p$. If we
generalize this, we find: 
\begin{equation}
a_{10} x_{10} = \overline{a_{10} x_{10}} + p\sum_{i = 1}^{a_{10} - 1} W\left( i
x_{10}, x_{10} \right).
\end{equation} 
Applying this for all terms in $f_0$ we find:
\begin{equation}
  f_1(x_{11}, \dots, x_{n 2}) = \sum_{i = 1}^{n}  (\overline{a_{i1} x_{i0}} +
  \overline{a_{i 0} x_{i
  1}}) - b_1 + \sum_{j = 1}^{n} \sum_{k = 1}^{a_{j 0} - 1}   W(k x_{j 0}, x_{j
  0}).
\end{equation} 
This way, all the multiplications are smaller than $p$, the sum of these
multiplications can still be $\ge p$. As an example, if we have $\overline{a} +
\overline{b} + \overline{c}$, where $\overline{a}, \overline{b}, \overline{c}
\in \mathbb F_p$, then 
\begin{equation}
  \begin{split}
    \overline{a} + \overline{b} + \overline{c} &= \overline{a + b} + \overline{c} +
    W(\overline{a}, \overline{b}) \\
    &= \overline{a + b + c} +
    W(\overline{a}, \overline{b}) +
    W(\overline{a + b}, \overline{c}).
  \end{split}
\end{equation}
Adding this to $f_1$ yields:
\begin{equation}
  f_1(x_{11}, \dots, x_{n 2}) = \sum_{i = 1}^{n}  (\overline{a_{i1} x_{i0}} +
  \overline{a_{i 0} x_{i
  1}}) - b_1 + \sum_{j = 1}^{n} \sum_{k = 1}^{a_{j 0} - 1}   W(k x_{j 0}, x_{j
  0}) + \sum_{j = 2}^{n} W \left( \sum_{k = 1}^{j - 1} a_{k 0} x_{k 0}, a_{j
  0} x_{j 0} \right).
\end{equation} 
Now, we have $f_0$ and $f_1$ and we know the following:

\begin{equation}
  \begin{split}
    a_1 x_1 + \dots + a_n x_n \neq b &\Leftrightarrow f_0(x_{11}, 
    \dots, x_{n 2}) + p f_1(x_{11}, 
    \dots, x_{n 2}) 
    \neq 0 \\
  \end{split}.
\end{equation} 
Then we use the same trick as in Task 2, namely applying Fermat's Little
Theorem:
\begin{equation}
\begin{cases}
  g_0 = (f_0 - b_0)^{p - 1} \\
  g_1 = (f_1 - b_1)^{p - 1} \\
\end{cases}
\end{equation} 
both of which are either $0$ or $1$, but they cannot be 0 at the same, otherwise
the condition above would no longer hold. It is however possible that in the
special case of $p = 2$, both $g_0 = 1$ and $g_1 = 1$, such that their sum is $0
mod 2$. For that case, we again add a $W$ function. The final function then
becomes:
\begin{equation}
  f = \left( (f_0 - b_0)^{p - 1} + (f_1 - b_1)^{p-1} + W(f_1, f_2)
  \right)^{p-1} - 1.
\end{equation} 
Due to the fact that the sum within the outer parenthesis is non-zero, Fermat's
Little Theorem states that raising it to the power $p - 1$ makes it 1.
Subtracting by one then leads to the desired property  $f(s) = 0$ iff the
disequality holds.

\subsubsection*{Task 3.c}

Table~\ref{tab:timing_results_3c} shows the timing results for $p^2 = 4$ and
$p^2 = 9$ for $r$ ranging from 1 to 3. We see an opposite pattern w.r.t. Task 2,
as here, the larger $r$, the longer it takes to solve the system, which can be
explained by the fact that increasing $r$ leads to a very large system of
equations, which takes longer to solve.

\begin{table}[htpb]
  \centering
  \caption{Timing results for $n = 8$.}
  \label{tab:timing_results_3c}
  \begin{tabular}{cccc}
  
    \toprule
    $r$ & $p^2$ & Time taken [s]  & \# eqs.\\
    \midrule
    1 & 4 & 0.010 & 78\\
    \midrule
    1 & 9 & 0.310 & 299\\
    \midrule
    2 & 4 & 0.010 & 155 \\
    \midrule
    2 & 9 & 0.510 & 597 \\
    \midrule
    3 & 4 & 0.010 & 232 \\
    \midrule
    3 & 9 & 0.800 & 896 \\
    \bottomrule
  \end{tabular}
\end{table}

\subsection*{Task 3.e}

I would think solving the disequations over a field would be easier, as in a
field a multiplicative inverse is always defined, which is not the case in a
ring, such that operations on elements of the field become easier.

\subsection*{Task 4}

Table~\ref{tab:dreg} shows the results for the degree of regularity for
increasing values of $n$ and $m = C_N \cdot n$. We see that if  $n$ increases,
$d_\text{reg}$ will also increase, so if we extrapolate,  $d_\text{reg}$ would
tend to $\infty$. This indicates that the larger $n$, the
more difficult it is to solve the problem, which can also be seen based on $m$,
the number of equations. The complexity of the system is then given by the
theory as 

\begin{equation}
  \mathbcal{O}\left( \left( m {{n + d_\text{reg} - 1}\choose{d_\text{reg}}}
  \right)^\omega  \right) .
\end{equation} 

For the exhaustive search, there would be $p^n$ amount of different secrets
available. The comparison in complexity can be seen in the table.

\begin{table}[h!]
\centering
\caption{Degree of regularity for $n$ and $m$.}
\begin{tabular}{ccccc}
\toprule
$n$ & $m$ & $d_\text{reg}$ & Complexity & $3^n$\\
\midrule
10  & 27   & 3  & $\mathcal{O}\left( 2.10e{11} \right)  $& 5.9e4\\
\midrule
110 & 298  & 11 & $\mathcal{O}\left( 4.14e52 \right)  $& 3.04e52\\
\midrule
210 & 569  & 18 & $\mathcal{O}\left( 1.48e 87 \right)  $& 1.57e100\\
\midrule
310 & 840  & 24 &$\mathcal{O}\left( 8.03e117 \right)  $ & 8.08e147\\
\midrule
410 & 1111 & 30 & $\mathcal{O}\left( 2.33e 148 \right)  $& 4.17e195 \\
\midrule
510 & 1382 & 37 & $\mathcal{O}\left( 1.61 e 182 \right)  $& 2.15e243\\
\midrule
610 & 1653 & 43 & $\mathcal{O}\left( 3.19e 212 \right)  $& 1.11e291 \\
\midrule
710 & 1924 & 49 & $\mathcal{O}\left( 5.57e 242 \right)  $& Inf \\
\midrule
810 & 2195 & 55 & $\mathcal{O}\left( 8.91e 272 \right)  $& Inf \\
\midrule
910 & 2466 & 61 & $\mathcal{O}\left( 1.3425e 303 \right)  $& Inf \\
\end{tabular}
\label{tab:dreg}
\end{table}

\clearpage

\subsection*{Task 5}

\subsubsection*{Task 5.a}
Our method is to calculate the expected amount of monomials in $\mathbb{F}_p$ of
degree $p - 1$, which is then the expected amount of disequations we need, as
every disequation will give rise to a linear equation and we need a full rank
matrix to solve that problem. The estimation lies in the fact that it's possible
that some disequations give rise to linear dependent equations, in which case we
need more disequations. The amount of monomials is given by the `stars and bars'
method. Using that method, we find that if we have $n$ variables and $m$ powers,
the amount of monomials is given by ${m + n - 1}\choose{n - 1}$. Now we want to know
the amount of monomials of degree $p-1$ or \textit{less}, so we have to take the
sum:
\begin{equation}
  \sum_{i = 1}^{p-1}  {{n + i - 1}\choose{n - 1}}.
\end{equation} 
We start from $i = 1$, as we're not counting the monomial 1, which is handled
separately.

If the solution from the linear system is not unique, we have to do extra work
to verify which of the solutions is the actual solution of the system of
disequations.

\subsubsection*{5.c}

The table below shows the results for $n = 8$ :

 \begin{table}[htpb]
  \centering
  \label{tab:label}
  \begin{tabular}{ccc}
    \toprule
    $p$ & Expected \# eqs. & Needed \# eqs. \\
    \midrule
    3 & 65 & 66 \\
    \midrule
    5 &1000&  1001 \\
    \midrule
    7 &8007& 8008\\
    \bottomrule
  \end{tabular}
\end{table}

We see that the total number of needed equations is almost the same as the
expected number of equations calculated above. The discrepancy can be explained
by the fact that by chance, some linearized equations can be linearly dependent.

\subsection*{Task 6}

For proof of completeness, we have to show that when both Peggy and Victor
follow the protocol, the \texttt{Verify} of Victor will always return
\texttt{true}. We have two cases: $c = 0$ and $c = 1$: 
 \begin{itemize}
   \item $c = 0$: then the response is $C$. By design, $M = C \cdot A \cdot
     C^{\text{trunc}(n)}$, so the verify will always return  \texttt{true}.
   \item $c = 1$: then the response is $\mathbf{m}$. We then find $M
     \cdot\mathbf{m} = C \cdot A \cdot C^{\text{trunc}(n)} \cdot \left(
     C^{\text{trunc}(n)} \right)^{-1} \mathbf{s} = C \cdot A \cdot \mathbf{s}
     \not\equiv \mathbf{0}$, as by design $C \cdot A \cdot \mathbf{s} \not\equiv
     \mathbf{0}.$
\end{itemize}

\subsection*{Task 8}

We first note that finding $\mathbf{t}$ is sufficient for finding $\mathbf{s}$
due to the Moore-Penrose inverse of $A$: 
\begin{equation}
  \mathbf{s} = A^+ A \mathbf{t},
\end{equation} 
with $A^+$ the pseudo-inverse. Secondly, we note that when receiving
$\mathbf{C}$, we actually get $m$ disequations, as $\mathbf{C} \cdot \mathbf{t}
\not\equiv \mathbf{0}$. 

Once we have enough disequations, we can all linearize them via the methods
described in Task 5, which will then give us a system of linear equations, which
is will yield $\mathbf{t}.$ As explained above, from  $\mathbf{t}$, we can
derive $\mathbf{s}$. The method from Task 5 only works when $N$ is prime.

\subsection*{Task 9}

The most important point to see is that for a signature to succeed, the
following relations between $A, M, C$ and $\mathbf{m}$ should hold:

\begin{equation}
\begin{cases}
  M = C \cdot A \cdot C^{\text{trunc}(n)} \\
  M \cdot \mathbf{m} \not\equiv \mathbf{0}
\end{cases}.
\end{equation} 

We note that neither of the two conditions requires $\mathbf{s}$, explicitly,
which makes it possible to attack the scheme. The most difficult problem is
finding an $\mathbf{m}$, given an $M$, so we evade that by choosing an $M$ and
$\mathbf{m}$ such that $M \cdot \mathbf{m} \not\equiv \mathb{0}$.

Then the next step is finding the $C$ matrix such that the first condition
holds. This is also difficult, as the equation is non-linear due to the fact
that $C^{\text{trunc}(n)}$ is derived from  $C$. We can make this equation
easier by choosing $C^{\text{trunc}(n)} = I$, such that in the scenario that $m
= 3$ and $n = 2$, we get: 
\begin{equation}
  \begin{bmatrix}
    m_1 & m_2  \\
      m_3 & m_4 \\
      m_5 & m_6 \\
      \end{bmatrix}  = \begin{bmatrix} 1 & 0 & c_1 \\ 0 & 1 & c_2 \\ c_3 & c_3 &
      c_5\end{bmatrix} \begin{bmatrix} a_1 & a_2 \\ a_3 & a_4 \\ a_5 & a_6
      \end{bmatrix} \begin{bmatrix} 1 & 0 \\ 0 & 1 \end{bmatrix}.
\end{equation} 

Here we can see that we have $5$ unknowns in the $C$-matrix and $6$ unknowns in
the $M$-matrix, but that we only have 6 equations. We can solve that issue by
choosing some unknowns fixed values, such that we remain with 6 unknowns. This
is not generally solvable, as (1) the system must be solvable and (2) we're
working with a non-prime order of the ring, which means that not all elements
have an inverse. Furthermore, if we choose fixed values for unknowns, we give
precedence to the $m$-values, as that will make it easier to find the
$\mathbf{m}$ later.

This method is however not generalizable and due to time constraints, I did not
implement it.

\printbibliography

\end{document}
